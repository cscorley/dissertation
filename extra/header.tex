\usepackage{chapterbib}
\usepackage{morefloats}
\usepackage{xspace}
\usepackage{lscape}

\setlength\epigraphwidth{\textwidth}
\setlength\epigraphrule{0pt}

\definecolor{lightred}{RGB}{175,50,50}
\definecolor{lightgreen}{RGB}{0,150,0}
\definecolor{lightblue}{RGB}{50,50,175}

\lstdefinelanguage{diff}{%
  morecomment=[f][\color{lightblue}]{diff },
  morecomment=[f][\color{lightblue}]{index },
  morecomment=[f][\color{lightblue}]{@@},     % group identifier
  morecomment=[f][\color{lightred}]-,         % deleted lines
  morecomment=[f][\color{lightgreen}]+,       % added lines
  morecomment=[f][\color{lightblue}]{---}, % Diff header lines (must appear after +,-)
  morecomment=[f][\color{lightblue}]{+++},
}
\hyphenation{}

\newcommand{\attn}[1]{{\color{red}#1}}
\newcommand{\desc}[1]{{\emph{\color{blue}#1}}}
\newcommand{\todo}[1]{\colorbox{yellow}{\bf TODO\: #1}}
\newcommand{\needcite}{\attn{\small{[cite]}}}

%\newcommand{\rqem}[1]{\emph{#1}}
\let\rqem\emph%

\definecolor{lightergray}{gray}{0.90}
\newcommand{\myrowcolor}{\rowcolor{lightergray}}

\newcommand{\frp}{\textit{Research Problem 1} (RP1)\xspace}
\newcommand{\frps}{RP1\xspace}
\newcommand{\frph}{RP1: Feature Location\xspace}

\newcommand{\fonep}{RQ 1.1\xspace}
\newcommand{\foneq}{Is a changeset-based FLT as accurate as a snapshot-based FLT?\xspace}
\newcommand{\fone}{\rqem{\fonep}\xspace}
\newcommand{\ftwop}{RQ 1.2\xspace}
\newcommand{\ftwoq}{Does the accuracy of a changeset-based FLT fluctuate as a project evolves?\xspace}
\newcommand{\ftwo}{\rqem{\ftwop}\xspace}


\newcommand{\drp}{\textit{Research Problem 2} (RP2)\xspace}
\newcommand{\drps}{RP2\xspace}
\newcommand{\drph}{RP2: Developer Identification\xspace}

\newcommand{\donep}{RQ 2.1\xspace}
\newcommand{\doneq}{Is a changeset-based DIT as accurate as a snapshot-based DIT?\xspace}
\newcommand{\done}{\rqem{\donep}\xspace}
\newcommand{\dtwop}{RQ 2.2\xspace}
\newcommand{\dtwoq}{Does the accuracy of a changeset-based DIT fluctuate as a project evolves?\xspace}
\newcommand{\dtwo}{\rqem{\dtwop}\xspace}


\newcommand{\crp}{\textit{Research Problem 3} (RP3)\xspace}
\newcommand{\crps}{RP3\xspace}
\newcommand{\crph}{RP3: Combining and Configuring Changeset-based Topic Models\xspace}

\newcommand{\conep}{RQ 3.1\xspace}
\newcommand{\coneq}{Can we use the same topic model in more than one context effectively?\xspace}
\newcommand{\cone}{\rqem{\conep}\xspace}
\newcommand{\ctwop}{RQ 3.2\xspace}
\newcommand{\ctwoq}{What are the effects of using different portions of
a changeset for corpus construction, such as added, removed, context lines, and
the commit message?\xspace}
\newcommand{\ctwo}{\rqem{\ctwop}\xspace}


\newcommand{\rrp}{\textit{Research Problem 4} (RP4)\xspace}
\newcommand{\rrps}{RP4\xspace}
\newcommand{\ronep}{RQ 4.1\xspace}
\newcommand{\roneq}{What are the effects of different pseudo-random seeds on an FLT?\xspace}
\newcommand{\rone}{\rqem{\ronep}\xspace}
\newcommand{\rtwop}{RQ 4.2\xspace}
\newcommand{\rtwoq}{What are the effects of different pseudo-random seeds on an DIT?\xspace}
\newcommand{\rtwo}{\rqem{\rtwop}\xspace}

\newcommand{\system}[1]{#1\xspace} % no formatting for now
\newcommand{\bookkeeper}{\system{BookKeeper v4.3.0}}
\newcommand{\mahout}{\system{Mahout v0.10.0}}
\newcommand{\openjpa}{\system{OpenJPA v2.3.0}}
\newcommand{\pig}{\system{Pig v0.14.0}}
\newcommand{\tika}{\system{Tika v1.8}}
\newcommand{\zookeeper}{\system{ZooKeeper v3.5.0}}

\lstset{%
  basicstyle=\ttfamily\footnotesize,        % the size of the fonts that are used for the code
  breakatwhitespace=false,         % sets if automatic breaks should only happen at whitespace
  breaklines=true,                 % sets automatic line breaking
  columns=fixed,
  keepspaces=true,                 % keeps spaces in text, useful for keeping indentation of code (possibly needs columns=flexible)
  numbers=left,                    % where to put the line-numbers; possible values are (none, left, right)
  numbersep=10pt,                   % how far the line-numbers are from the code
  numberstyle=\tiny, % the style that is used for the line-numbers
  showspaces=false,                % show spaces everywhere adding particular underscores; it overrides 'showstringspaces'
  showstringspaces=false,          % underline spaces within strings only
  showtabs=false,                  % show tabs within strings adding particular underscores
  stepnumber=1,                    % the step between two line-numbers. If it's 1, each line will be numbered
  tabsize=2,                       % sets default tabsize to 2 spaces
}

\urlstyle{rm}

%\BeforeBeginEnvironment{lstlisting}{\begin{singlespace}}
%\AfterEndEnvironment{lstlisting}{\end{singlespace}\noindent\ignorespaces}
