\section{Approach}\label{approach}

\begin{enumerate}
\def\labelenumi{\arabic{enumi}.}
\itemsep1pt\parskip0pt\parsep0pt
\item
  Build model from changesets
\item
  \emph{Do not} infer a $\theta_{changesets}$
\item
  Infer a $\theta_{files}$ and $\theta_{developers}$
\item
  On a new commit, repeat step 2 \emph{only on the changed document}
\end{enumerate}

\section{Primary studies}\label{primary-studies}

\subsection{Feature location}\label{feature-location-study}

\subsection{Bug localization}\label{bug-localization-study}

\subsection{Developer
identification}\label{developer-identification-study}

\subsubsection{Introduction}\label{introduction}

Software features are functionalities that are defined by requirements
and are accessible to developers and users. Software change is
continual, because new features must be added to meet revised
requirements, existing features must be enhanced to satisfy increased
expectations, and defective features (i.e., bugs) must be removed to
achieve intended behavior. Changes to a software system are proposed by
developers or users, who submit change requests to the project issue
tracker. Change requests are sometimes called issue reports, and
specific kinds of change requests include feature requests, enhancement
requests, and bug reports.

Triaging a change request involves several steps that can be completed
either by a single person or by a team of developers in a triage
meeting. How triage occurs differs from team to team, but the general
steps required are as follows. First, the triager(s) must see if the
request has enough information to be considered. It is marked as a
duplicate if the request already exists. After it is confirmed that the
request is new and has enough information, a decision must be made if
and how soon it will be completed based on its severity, frequency,
risk, and other factors. Finally, the triager assigns a request to the
developer. Ultimately, the goal of triage is deciding priority of the
request and assignment to the developer that is best suited to complete
the change request.

Triaging is a common and difficult task. Triage is even more difficult
on projects where developer teams are large or geographically
distributed\scite{Herbsleb2001Empirical}. A project member triaging a
change request will need to consider several factors in order to
correctly assign the change request to a set of developers with
appropriate expertise\scite{McDonald1998Just}. Triaging requires
contextual knowledge about the product, team structure, individual
expertise, workload balance, and development schedules in order to
correctly assign a change request.

Triaging can be a time consuming and error prone process when done
manually. If a change request was assigned in error, it will need to be
reassigned to the appropriate developer. Jeong et
al.\scite{Jeong2009Improving} found that reassignment occurs between
37\%-44\% of the time and introduces an average of 50 days delay in
completing the request. Automated support for triaging helps to decrease
change request time-to-triage and to correct, or prevent, human error.

McDonald and Ackerman\scite{McDonald1998Just} show that there are two
expertise finding problems: identification and selection. In a
semi-automated system, expertise identification is automated, and
suggests an expert for selection. In a fully-automated system, the
expert is identified and selected for assignment to the change request.
Anvik\scite{Anvik2006Automating} notes that a fully-automated approach
may not be feasible given the amount of contextual knowledge required
for triage.

\section{Supporting studies}\label{supporting-studies}

\subsection{Corpora of software
histories}\label{corpora-of-software-histories}

\subsection{Implementing topic models with growing
vocabulary}\label{implementing-topic-models-with-growing-vocabulary}
