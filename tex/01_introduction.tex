Software developers are often confronted with maintenance tasks that
involve navigation of repositories that preserve vast amounts of project
history. Navigating these software repositories can be a time-consuming
task, because their organization can be difficult to understand.
Fortunately, topic models such as \abbr{LDA}{latent Dirichlet
allocation}\scite{Blei-etal:2003} can help developers to navigate and
understand software repositories by discovering topics (word
distributions) that reveal the thematic structure of the
data\scite{Linstead-etal:2007,Thomas-etal:2011,Hindle-etal:2012}.

Program comprehension is a prerequisite to incremental change. A
software developer who is tasked with changing a large software system
spends effort on program comprehension activities to gain the knowledge
needed to make the change\scite{Corbi:89}. For example, the developer
spends effort to understand the system architecture or to locate the
parts of the source code that implement the feature(s) being changed.
Gaining such knowledge can be a time-consuming task, especially for
developers who are unfamiliar with the system. Topic models of source
code can help such developers to understand the system by revealing a
latent structure that is not obvious from the package hierarchy or
system documentation\scite{Savage-etal:10}.

Topic models are clusters of source code entities (e.g., classes) that
are grouped by their natural language content (i.e., the words in their
identifiers, comments, and literals). Such topics often correspond to
the concepts and features implemented by the source
code\scite{Baldi-etal:08}, and exploring such topics shows promise in
helping developers to understand the entities that make up a system and
to understand how those entities
relate\scite{Kuhn-etal:07,Maskeri-etal:08,Savage-etal:10,Gethers-etal:11a}.
Recent approaches to exploring linguistic topics in source code use
\abbr{ML}{machine learning} techniques that model correlations among
words, such as \abbr{LSI}{latent semantic
indexing}\scite{Deerwester-etal:90} and \abbr{LDA}{latent Dirichlet
allocation}\scite{Blei-etal:03}, and ML techniques that also model
correlations among documents, such as RTM\scite{Chang-Blei:10}.

Topic models of source code have many applications in addition to
general program comprehension. These applications include feature
location\needcite, bug localization\scite{Rao-etal:13}, triaging
incoming change requests\scite{Kagdi-etal:11}, aspect
mining\scite{Baldi-etal:08}, and traceability link
recovery\scite{Asuncion-etal:10}. Yet, while researchers have had
success in using topic models on source code entities, there is a
fundamental issue with the current approaches. This issue is that the
input documents used to build a topic model are source code entities,
and will be the motivating point of this work.

\section{Motivation}\label{motivation}

When modeling a source code repository, the corpus typically represents
a snapshot of the code. That is, a topic model is often trained on a
corpus that contains documents that represent files from a particular
version of the software. Keeping such a model up-to-date is expensive,
because the frequency and scope of source code changes necessitate
retraining the model on the updated corpus. However, it may be possible
to automate certain maintenance tasks without a model of the complete
source code. For example, when assigning a developer to a change task, a
topic model can be used to associate developers with topics that
characterize their previous changes. In this scenario, a model of the
changesets created by each developer may be more useful than a model of
the files changed by each developer. Moreover, as a typical changeset is
smaller than a typical file, a changeset-based model is less expensive
to keep current than a file-based model.

While using file-based models is a natural fit for program comprehension
tasks such as feature location and bug localization, they still are
unable to stay up-to-date entirely. Additionally, much of the work for
assigning developers to change requests still uses files as input and an
array of heuristics to identify a
developer\scite{Kagdi-etal:11}\needcite. These methods also have the
same flaw in that they ultimately rely on files for information.

Like Rao et al.\scite{Rao-etal:13}, the motivation of this work is to
create topic models that can be incrementally updated over time.
However, unlike Rao et al., we can rely on the source code history
itself to build the model without needing to manually adjust model
latent variables. This gains the benefit of an increase in query time,
but also could lead to a more reliable model.

\section{Research goals}\label{research-goals}

The primary research goal of this proposal is to evaluate the
performance and reliability of topic models built on the source code
histories. This will require configuring and executing studies in
various contexts of software maintenance work, such as feature location,
bug localization, and developer identification.

A secondary goal is to create a practical framework for building models
that can be used in multiple contexts. This will require building a
prototype tool that could be used by both researchers and practitioners.

\section{Outline}\label{outline}

In this proposal we propose an approach towards building practical,
online topic models for automating software maintenance tasks. In
Chapter\sref{ch:related} we discuss the background and related works.
Chapter\sref{ch:previous} covers previous work already achieved towards
the research goals. Chapter\sref{ch:proposed} outlines the primary
studies and their evaluations, along with supporting studies. A
projected schedule for completion of these studies is given in
Chapter\sref{ch:schedule}. Finally, we conclude this proposal in
Chapter\sref{ch:conclusion}.
